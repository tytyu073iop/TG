\documentclass[a4paper, 12pt]{article}
\usepackage[english, russian]{babel}
\usepackage{listings}
\usepackage[top=2cm, bottom=2cm, left=2cm, right=2cm, bindingoffset=0pt]{geometry}
\usepackage{hyperref}
\usepackage{amsmath,amsfonts,amssymb,amsthm,mathtools}
\usepackage[T2A]{fontenc}
\usepackage[utf8]{inputenc}
\newtheorem*{theorem*}{Теорема}
\theoremstyle{definition}
\newtheorem*{definition*}{Определение}
\renewcommand\qedsymbol{$\blacksquare$}
\usepackage{tikz}
\usetikzlibrary{positioning}
\usetikzlibrary {graphs}

\renewcommand{\leq}{\leqslant}
\renewcommand{\geq}{\geqslant}

\title{Типы уравнений}
\author{Бирюк Илья Александрович}

\begin{document}
  \maketitle
  \newpage
  \tableofcontents
  \newpage
  \section{Name}
  \begin{theorem*}
    Пусть $G$ -- это $(n,m)$ граф, $k$ -- число компонент связности
    Тогда
    $$n-k\leq{m}\leq\frac{(n-k)(n-k+1)}{2},.$$
  \end{theorem*}
  
  \begin{proof}
    $m\leq{n-k}$ - Доказывается по мат индукции
    
    $m\leq\frac{(n-k)(n-k+1)}{2}$ - 
    
    Берём $k\geq{2}$
    
    \begin{enumerate}
      \item рисуем $k$ полных графов
      
      \begin{tikzpicture}[main/.style = {draw, circle}]
        \node[main](1){$G_1$};
        \node[main](2)[right of=1]{$G_2$};
        \node (dots) [right of=2] {$\cdots$};
        \node[main](k)[right of=dots]{$G_k$};
      \end{tikzpicture}
      \item Вынимаем из $G_{k-1}$ точку и перемещаем её в $G_k$ (сохраняя полноту). Возьмём, что $\forall n\le k, V(G_k)\geq V(G_n)$. Тогда количество рёбер изменится на $V(G_k)-(V(G_n)-1) > 0$.
      \item Повторяем так, пока все кроме последнего подграфа не будут тривиальными (то есть пока они не будут иметь одну вершину).
      \item Самый экстремальный случай, изолированные вершины и $K_{n-k+1}$, тогда число рёбер $$C^2_{n-k+1}=\frac{(n-k)(n-k+1)}{2}.$$ 
    \end{enumerate}
  \end{proof}
  
  \begin{theorem*}
    Пусть $G$ связный граф и $e\in E(G)$.
    
    \begin{enumerate}
      \item Если $е$ принадлежит некоторому циклу, то граф $G - e$ связен
      \item Если $е$ не принадлежит никакому циклу, то граф $G - e$ содержит ровно $2$ компоненты связности
    \end{enumerate}
  \end{theorem*}
  \begin{proof}
    Возьмём $e=uv,e\in E(G)$
    
    \begin{enumerate}
      \item Если $е$ принадлежит некоторому циклу, то граф $G - e$ связен
      
      \begin{enumerate}
        \item Нарисуем цикл
        
        \begin{tikzpicture}[graph/.style = {draw, circle}, dot/.style = {draw, circle}]
          \node[dot](1){};
          \node[dot](-1)[below right of=1]{};
          \node[dot](2)[above right of=1]{};
          \node (dots) [right of=2] {$\cdots$};
          \node[dot](u)[right of=dots]{$u$};
          \node (dotsb) [right of=-1] {$\cdots$};
          \node[dot] (v) [right of=dotsb] {$v$};
          \draw (1) -- (2);
          \draw (1) -- (-1);
          \draw (-1) -- (dotsb);
          \draw (dotsb) -- (v);
          \draw (2) -- (dots);
          \draw (dots) -- (u);
          \draw (u) -- node[anchor=west]{$e$} (v) ;
        \end{tikzpicture}
        \item Удалим $e$
        
        \begin{tikzpicture}[graph/.style = {draw, circle}, dot/.style = {draw, circle}]
          \node[dot](1){};
          \node[dot](-1)[below right of=1]{};
          \node[dot](2)[above right of=1]{};
          \node (dots) [right of=2] {$\cdots$};
          \node[dot](u)[right of=dots]{$u$};
          \node (dotsb) [right of=-1] {$\cdots$};
          \node[dot] (v) [right of=dotsb] {$v$};
          \draw (1) -- (2);
          \draw (1) -- (-1);
          \draw (-1) -- (dotsb);
          \draw (dotsb) -- (v);
          \draw (2) -- (dots);
          \draw (dots) -- (u);
        \end{tikzpicture}
        
        Как можно заметить, не появилось ни одной компоненты связности.
      \end{enumerate}
      \item Если $е$ не принадлежит никакому циклу, то граф $G - e$ содержит ровно 2 компоненты связности
      \begin{enumerate}
        \item Учитывая условия выше, мы можем разделить граф на 2 части, имеющие маршрут к $u$ без $v$ и наоборот
        
        \begin{tikzpicture}[graph/.style = {draw, circle}, dot/.style = {draw, circle}]
            \node[graph](1){$G_u$};
            \node[dot](u)[right of=1]{$u$};
            \draw (1) -- (u);
            \node[dot](v)[right =0.5cm of u]{$v$};
            \draw (u) -- node[anchor=south]{$e$} (v);
            \node[graph](2)[right of=v]{$G_v$};
            \draw (v) -- (2);
        \end{tikzpicture}
        \item Удаляем ребро $e$, и видим, что появилось 2 компоненты связности
        
        \begin{tikzpicture}[graph/.style = {draw, circle}, dot/.style = {draw, circle}]
            \node[graph](1){$G_u$};
            \node[dot](u)[right of=1]{$u$};
            \draw (1) -- (u);
            \node[dot](v)[right =0.5cm of u]{$v$};
            \node[graph](2)[right of=v]{$G_v$};
            \draw (v) -- (2);
        \end{tikzpicture}
      \end{enumerate}
    \end{enumerate}
  \end{proof}
  
  \section{Метрические характеристики графов}
  Для параграфа: G - связен

  \textbf{Определение.} Расстояние $d(u,v)$ между вершинами $u\neq v$ графа G --  \textit{длинна кратчайшей простой цепи}, если $u=v$, то $d(u,v)=0$
  
  \textbf{Свойства:}
  \begin{enumerate}
    \item \textbf{Свойство неотрицательности.}
    $$d(u,v)\geq 0\ \text{и}\ d(u,v) = 0 \Leftrightarrow u=v,\ \forall u,v\in V(G).$$
    \item \textbf{Свойство симметрии.} 
    $$d(u,v)=d(v,u),\ \forall u,v\in V(G).$$ 
    \item \textbf{Свойство треугольников.}
    $$d(u,v)\leq d(u,w)+d(w,u),\ \forall u,v\in V(G).$$
  \end{enumerate}
  
  \textbf{Определение.} \textit{Эксцентриситетом вершины} называется величина $$e(v)=\max d(v,u),v\in V(G),$$ то есть максимальное расстояние от вершины до другой какой-либо вершины графа).
  
  \textbf{Определение.} \textit{Радиусом графа} называется величина $$r(G)=\min e(v),\ v\in V(G).$$
  
  \textbf{Определение.} \textit{Диаметром графа} называется величина $$d(G)=\max e(v),\ v\in V(G).$$
  
  \textbf{Определение.} \textit{Вершина в графа $G$ называется \textit{центральной}, если $e(v)=r(G)$ и \textbf{периферической}, если $e(v)=d(G)$.}
  
  \textbf{Определение.} Центр графа, множество всех его центральных вершин, перефирия, перефирийных.
  
  Пример, в круге Эксцентриситет вершины:

  \begin{tikzpicture}[graph/.style = {draw, circle}, dot/.style = {draw, circle}]
    \node[dot](1){4};
    \node[dot](2)[above right of=1]{3};
    \node[dot](3)[right of=2]{2};
    \node[dot](4)[right of=3]{3};
    \node[dot](5)[below right of=4]{4};
    \draw (1) -- (2) -- (3) -- (4) -- (5);
  \end{tikzpicture}
  $r(P_5)=2, d(P_5)=4$
  
  \begin{theorem*}
    Для любого графа $H$ существует граф $G$, центр которого порождает $H$.
  \end{theorem*}
  \begin{proof}
    \begin{enumerate}
        \item Возьмём граф $H$
        
        \begin{tikzpicture}[graph/.style = {draw, circle}, dot/.style = {draw, circle}]
            \node[graph](H){$H$};
        \end{tikzpicture}
        \item Добавим к нему вершины $x,y,z,t$, $x$ и $y$ Соедениены со всеми вершинами $H$
        
        \begin{tikzpicture}[graph/.style = {draw, circle}, dot/.style = {draw, circle}]
            \node[graph](H){$H$};
            \node[dot](x)[left =0.5cm of H]{$x$};
            \node[dot](y)[right =0.5cm of H]{$y$};
            \node[dot](z)[below of=x]{$z$};
            \node[dot](t)[below of=y]{$t$};
            \draw (z) -- (x) (y) -- (t);
            \draw[double] (x) -- (H) -- (y);
        \end{tikzpicture}

        Как видно $\forall v \in V(H), e(v)=r(G)=2$
    \end{enumerate}
  \end{proof}
  \begin{theorem*}
  Для любого связного графа ж верно: $r(G)\leq diam(G)\leq 2r(G)$
  \end{theorem*}
  \begin{proof}
    \begin{enumerate}
        \item $r(G)\leq diam(G)$ - очевидно
        \item $diam(G)\leq 2r(G)$. Берём две переферичиские$(u,v)$ и одну центральную$(w)$. Тогда данное равенство получается через равенство треугольника:
        
        \begin{tikzpicture}[graph/.style = {draw, circle}, dot/.style = {draw, circle}]
            \node[dot](w){$w$};
            \node[dot](u)[below left of=w]{$u$};
            \node[dot](v)[below right of=w]{$v$};
            \draw (u) -- (w) -- (v);
            \draw[dotted] (u) -- (v);
        \end{tikzpicture}
    \end{enumerate}
  \end{proof}
\end{document}
