\documentclass{article}
\usepackage[english, russian]{babel}
\usepackage{listings}
\usepackage[a4paper, top=2cm, bottom=2cm, left=2cm, right=2cm]{geometry}
\usepackage{hyperref}
\usepackage{amsmath}

\title{Типы уравнений}
\author{Бирюк Илья Александрович}

\begin{document}
\maketitle
\newpage
\tableofcontents
\newpage
\section{}
Теорема: Пусть G-(n,m) граф, k - число компонент связности

Тогда: $n-k\leq{m}\leq\frac{(n-k)(n-k+1)}{2}$

Доказательство:

$m\leq{n-k}$ - Доказывается по мат индукции

$m\leq\frac{(n-k)(n-k+1)}{2}$ - 

Нарисовать 1

Берём $k\geq{2}$



Самый экстримальный случай, изолированные вершины и $K_{n-k+1}$, тогда число рёбер $C^2_{n-k+1}=\frac{(n-k)(n-k+1)}{2}$ 

Теорема: Пусть G связный граф и $e\in E(G)$

\begin{enumerate}
    \item Если е принадлежит некоторому циклу, то G - e связен
    \item Если е не, то G-e содержит ровно 2 компоненты связности
\end{enumerate}
Доказательство:
\begin{enumerate}
    \item $e=uv\in E(G)$
    \item Разорвём связь, тогда наш граф разделим на две части $u\in G_u$ и $v\in G_v$$x\in V(G)$
\end{enumerate}
\section{Метрические характеристики графов}
Для параграфа: G - связен

Определение: Расстояние $d(u,v)$ между вершинами $u\neq v$ графа G - длинна кратчайшей простой цепи, если $u=v$, то $d(u,v)=0$

Свойства:
\begin{enumerate}
    \item $d(u,v)\geq 0$ и $d(u,v) = 0 \leftrightarrow u=v, \forall u,v\in V(G)$ - Свойство неотррицательности
    \item $d(u,v)=d(v,u), \forall u,v\in V(G)$ Свойство симметрии
    \item $d(u,v)\leq d(u,w)+d(w,u), \forall u,v\in V(G)$ - свойство треугольников
\end{enumerate}

Определение: Экстриситент вершины - $e(v)=max d(v,u),v\in V(G)$ (максимальное расстояние от вершины до другой какой-либо вершины графа)

Определение: радиус графа - $r(G)=\min e(v),v\in V(G)$

Определение: диаметр графа - $d(G)=\max e(v),v\in V(G)$

Определение: Вершина в графа ж называется центральной, если $e(v)=r(G)$ и перефирической, если $e(v)=d(G)$
Определение: Центр графа, множество всех его центральных вершин, перефирия, перефирийных.

Пример 2

Теорема: Для любого графа н существует граф ж, центр которого порождает н.
Доказательство: 3
Теорема: Для любого связного графа ж верно: $r(G)\leq diam(G)\leq 2r(G)$
Доказательство: $r(G)\leq diam(G)$ - очевидно
$diam(G)\leq 2r(G)$
4
\end{document}